\section{Bài tập và Code}

\subsection{Bài 1} Viết một đoạn code tính tổng Riemann trung tâm cho hàm số f(x,y) trên miền hình chữ nhật $R$ \\
(a) Input: $f(x,y)$, $a$, $b$, $c$, $d$ để xác định $R$, $m$, $n$ là số đoạn chia của [$a$, $b$] và [$c$, $d$]. \\
(b) Output: Tổng Riemann. \\

Đoạn code (Matlab):
\begin{lstlisting}[language=matlab]
    syms x y;
    f = input('Nhap ham so f(x,y): ');
    a = input('Nhap gia tri a: ');
    b = input('Nhap gia tri b: ');
    c = input('Nhap gia tri c: ');
    d = input('Nhap gia tri d: ');
    m = input('Nhap so doan chia cua x: ');
    n = input('Nhap so doan chia cua y: ');
    delta_x = (b - a) / m;
    delta_y = (d - c) / n;
    sum_Riemann = 0;
    for i = 1:m
        for j = 1:n
            x_val = a + (i - 0.5) * delta_x;
            y_val = c + (j - 0.5) * delta_y;
            sum_Riemann = sum_Riemann + double(subs(f, [x, y], [x_val, y_val])) * delta_x * delta_y;
        end
    end
fprintf("Tong Riemann giua la: %f\n", sum_Riemann);
\end{lstlisting}