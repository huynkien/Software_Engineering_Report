\subsection{Thiết lập lịch trình cho sinh viên}

\begin{figure}[H]
    \centering
    \includegraphics[width=1\linewidth]{graphics/DevelopmentView/Scheduling/ComponentSchedule.png}
    \caption{Sơ đồ thành phần cho use-case "Xem lại lịch học" cho sinh viên}
    \label{fig:componentscheduling}
\end{figure}

\subsection*{Mô tả sơ đồ thành phần: Xem lại lịch học}
Hình~\ref{fig:componentscheduling} mô tả kiến trúc thành phần của chức năng
\emph{Quản lý lịch học} ở góc nhìn \emph{development view}. 
Các thành phần được tổ chức thành ba tầng chính:
tầng trình bày (Presentation), các module nghiệp vụ (Change/Cancel/Review) 
và các thành phần lưu trữ (Persistent), cùng với module Thông báo.

\subsubsection*{Thành phần Presentation}

\begin{itemize}
    \item \textbf{Presentation}: đại diện cho giao diện người dùng của hệ thống.
    Nó cung cấp màn hình cho ba use-case: đổi lịch, hủy lịch và xem lại lịch.
    \item Presentation \emph{phụ thuộc} vào ba interface dịch vụ:
    \texttt{IChangeScheduleService}, \texttt{ICancelScheduleService} 
    và \texttt{IReviewScheduleService}. 
    Các quan hệ phụ thuộc này được thể hiện bằng các đường nét đứt từ Presentation
    đến ba vòng tròn interface tương ứng.
    \item Đồng thời, Presentation cũng kết nối với \emph{Notification service} bên ngoài
    để nhận và hiển thị các thông báo liên quan đến lịch học cho sinh viên
    (ví dụ thông báo đổi lịch thành công, hủy lịch thành công).
\end{itemize}

\subsubsection*{Các module nghiệp vụ}

\begin{itemize}
    \item \textbf{Change Module}:
    \begin{itemize}
        \item Hiện thực interface \texttt{IChangeScheduleService} 
        (thể hiện bằng socket ở thành phần Change Module gắn với ball của interface).
        \item Chịu trách nhiệm xử lý yêu cầu \emph{đổi lịch}: kiểm tra đăng ký hiện có,
        cập nhật ca học mới vào lịch và phát sinh sự kiện thông báo nếu cần.
        \item Sử dụng \textbf{Registration Persistent} để quản lý danh sách đăng ký lớp
        (connector ``Manage registrations'').
        \item Sử dụng \textbf{Schedule Persistent} để cập nhật thông tin lịch học 
        (connector ``Manage schedules'').
    \end{itemize}

    \item \textbf{Cancel Module}:
    \begin{itemize}
        \item Hiện thực interface \texttt{ICancelScheduleService}.
        \item Xử lý yêu cầu \emph{hủy lịch}: kiểm tra điều kiện hủy, 
        xác định lớp có được phép hủy hay không và nếu hợp lệ thì xóa lịch tương ứng.
        \item Kết nối với \textbf{Registration Persistent} (đọc các lớp đã đăng ký),
        \textbf{Condition Persistent} (đọc các quy tắc/thời hạn hủy -- connector ``Manage rules'')
        và \textbf{Schedule Persistent} (xóa mục lịch -- connector ``Manage schedules'').
        \item Các thao tác hủy thành công có thể phát sinh sự kiện tới 
        \textbf{Notification Module} thông qua interface \emph{Notification service}
        (đường nét đứt từ Cancel Module sang vòng tròn Notification service).
    \end{itemize}

    \item \textbf{Review Module}:
    \begin{itemize}
        \item Hiện thực interface \texttt{IReviewScheduleService}.
        \item Đảm nhiệm chức năng \emph{xem lại lịch học}: lấy lịch theo tuần hiện tại
        hoặc theo tiêu chí lọc (tuần/môn học).
        \item Sử dụng \textbf{Schedule Persistent} thông qua interface ``Manage schedules''
        để truy vấn dữ liệu lịch học.
    \end{itemize}
\end{itemize}

\subsubsection*{Các thành phần lưu trữ (Persistent)}

Mỗi thành phần Persistent tương ứng với một \emph{kho dữ liệu} riêng, 
cung cấp các interface truy cập dữ liệu cho các module nghiệp vụ:

\begin{itemize}
    \item \textbf{Registration Persistent}:
    \begin{itemize}
        \item Lưu trữ thông tin đăng ký lớp của sinh viên.
        \item Cung cấp interface \emph{Manage registrations} cho Change Module và Cancel Module
        đọc/ghi danh sách đăng ký.
    \end{itemize}
\newpage
    \item \textbf{Condition Persistent}:
    \begin{itemize}
        \item Lưu trữ các điều kiện/quy tắc áp dụng cho việc hủy hoặc đổi lịch 
        (thời hạn, trạng thái lớp, v.v.).
        \item Cung cấp interface \emph{Manage rules} cho Cancel Module để kiểm tra điều kiện hủy.
    \end{itemize}

    \item \textbf{Schedule Persistent}:
    \begin{itemize}
        \item Lưu trữ thông tin lịch học theo tuần, ca học, phòng học.
        \item Cung cấp interface \emph{Manage schedules} cho Change Module, Cancel Module
        và Review Module để đọc/ghi dữ liệu lịch.
    \end{itemize}

    \item \textbf{Notification Persistent}:
    \begin{itemize}
        \item Lưu trữ các thông báo đã gửi cho người dùng 
        (ví dụ thông báo đổi lịch/hủy lịch).
        \item Cung cấp interface \emph{Manage notifications} cho Notification Module.
    \end{itemize}
\end{itemize}

\subsubsection*{Notification Module và Notification service}

\begin{itemize}
    \item \textbf{Notification Module}:
    \begin{itemize}
        \item Nhận các yêu cầu gửi thông báo từ các module nghiệp vụ 
        thông qua interface \emph{Notification service} (ball–socket ở bên phải hình).
        \item Thực hiện logic gửi và lưu thông báo, 
        sử dụng \textbf{Notification Persistent} để 
        ghi nhận lịch sử thông báo (connector ``Manage notifications'').
    \end{itemize}
    \item Interface \emph{Notification service} cũng được kết nối tới 
    thành phần Presentation, cho phép UI đăng ký/nhận thông báo liên quan
    tới lịch học của sinh viên.
\end{itemize}

Nhìn chung, biểu đồ thành phần này cho thấy rõ:
\begin{itemize}[leftmargin=1.5cm]
    \item Presentation chỉ phụ thuộc vào các interface dịch vụ trừu tượng,
    không phụ thuộc trực tiếp vào lớp triển khai hay tầng lưu trữ.
    \item Mỗi module nghiệp vụ (Change, Cancel, Review) được tách riêng, 
    truy cập dữ liệu thông qua các Persistent tương ứng,
    giúp việc mở rộng hoặc thay đổi logic từng phần trở nên dễ dàng hơn.
    \item Cơ chế thông báo được gom vào một module độc lập (Notification Module), 
    nhận sự kiện từ các module lịch học và quản lý lưu trữ thông báo tập trung.
\end{itemize}


\newpage