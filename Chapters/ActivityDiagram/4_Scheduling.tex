\subsection{Thiết lập lịch trình cho sinh viên}
\begin{figure}[H]
    \centering
    \includegraphics[width=0.9\linewidth]{graphics/ActivityDiagrams/SchedulingActivity/DoiLich.png}
    \caption{Sơ đồ hoạt động cho use-case "Đổi lịch học" cho sinh viên}
    \label{fig:thietlaplichtrinhchosinhvien}
\end{figure}
\newpage
\subsection*{Mô tả sơ đồ hoạt động: Đổi lịch học}
\begin{itemize}
    \item \textbf{Người dùng} chọn chức năng \textbf{``Đổi lịch''} trên giao diện.
    \item \textbf{Hệ thống} hiển thị danh sách các lớp mà sinh viên đã đăng ký.
    \item \textbf{Người dùng} chọn lớp muốn đổi.
    \item \textbf{Hệ thống} hiển thị danh sách các ca học mới có sẵn cho lớp được chọn.
    \item \textbf{Người dùng} chọn ca học mới.
    \item \textbf{Hệ thống} tiến hành \textbf{kiểm tra trùng lịch} với các lớp khác trong thời khóa biểu hiện tại.
    \begin{itemize}
        \item Nếu \textbf{bị trùng lịch}: \textbf{Hệ thống} hiển thị thông báo lỗi ``Trùng lịch'' và yêu cầu người dùng chọn lại.
        \item Nếu \textbf{không trùng lịch}: \textbf{Hệ thống} cập nhật thay đổi và \textbf{gửi thông báo} xác nhận đổi lịch thành công.
    \end{itemize}
    \item \textbf{Người dùng} xác nhận thay đổi và kết thúc use-case.
\end{itemize}
\newpage
\begin{figure}[H]
    \centering
    \includegraphics[width=1\linewidth]{graphics/ActivityDiagrams/SchedulingActivity/HuyLich.png}
    \caption{Sơ đồ hoạt động cho use-case "Hủy lịch học" cho sinh viên}
    \label{fig:huylichhocchoSinhVien}
\end{figure}

\subsection*{Mô tả sơ đồ hoạt động: Hủy lịch học}
\begin{itemize}[leftmargin=1.5cm]
    \item \textbf{Người dùng} chọn chức năng \textbf{``Hủy lịch''}.
    \item \textbf{Hệ thống} hiển thị danh sách các lớp đã đăng ký.
    \item \textbf{Người dùng} chọn lớp muốn hủy.
    \item \textbf{Hệ thống} hiển thị thông tin chi tiết và các cảnh báo liên quan (ví dụ: thời hạn hủy, điều kiện học vụ).
    \item \textbf{Người dùng} xác nhận thao tác hủy.
    \item \textbf{Hệ thống} \textbf{kiểm tra điều kiện hủy} (thời hạn, trạng thái lớp, quy định nhà trường, \ldots).
    \begin{itemize}
        \item Nếu \textbf{không được phép hủy}: hệ thống hiển thị thông báo lý do không thể hủy.
        \item Nếu \textbf{được phép hủy}: hệ thống xóa lớp khỏi lịch học của sinh viên và thông báo hủy thành công.
    \end{itemize}
\end{itemize}

\begin{figure}[H]
    \centering
    \includegraphics[width=1\linewidth]{graphics/ActivityDiagrams/SchedulingActivity/XemLaiLich.png}
    \caption{Sơ đồ hoạt động cho use-case "Xem lại lịch học" cho sinh viên}
    \label{fig:xemlaichocchosinhvien}
\end{figure}

\subsection*{Mô tả sơ đồ hoạt động: Xem lại lịch học}
\begin{itemize}[leftmargin=1.5cm]
    \item \textbf{Người dùng} chọn chức năng \textbf{``Xem lại lịch''}.
    \item \textbf{Hệ thống} mặc định hiển thị \textbf{lịch học theo tuần hiện tại}.
    \item Nếu \textbf{người dùng} muốn xem lịch theo tuần khác hoặc lọc theo môn học, \textbf{người dùng} chọn \textbf{tuần/môn cần lọc}.
    \item \textbf{Hệ thống} áp dụng tiêu chí lọc và hiển thị \textbf{lịch học đã được lọc} tương ứng.
\end{itemize}

\newpage