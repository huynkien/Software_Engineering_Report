\subsection{Thông báo và nhắn tin}

\begin{figure}[H]
    \centering
    \includegraphics[width=1\linewidth]{graphics/figures/Chat.png}
    \caption{Sơ đồ use-case cho chức năng Thông báo và nhắn tin}
    \label{fig:chat}
\end{figure}
\newpage

\begin{longtable}{|p{0.2\linewidth}|p{0.75\linewidth}|}
    \caption{Bảng đặc tả Use-case: Xem tin nhắn và thông báo (UC-CHAT-01)}. \\
    \hline
    \textbf{Use-case} & \textbf{UC-CHAT-01:} Xem tin nhắn và thông báo. \\
    \hline
    \textbf{Actor} & Người dùng (Sinh viên/Tutor). \\
    \hline
    \textbf{Descriptions} & Sinh viên có thể xem thông báo/tin nhắn từ Tutor/Nhà trường; Tutor có thể xem tin nhắn từ sinh viên và thông báo từ Nhà trường. \\
    \hline
    \textbf{Include} & Không \\
    \hline
    \textbf{Extend} & \textbf{UC-CHAT-02:} Tìm kiếm tin nhắn. \\
    \hline
    \textbf{Precondition} & Người dùng đã đăng nhập thành công. \\
    \hline
    \textbf{Postcondition} & Hiển thị danh sách các tin nhắn và thông báo từ người dùng/Nhà trường. \\
    \hline
    \textbf{Trigger} & Người dùng chọn icon "Tin nhắn" ở trang chính. \\
    \hline
    \textbf{Normal Flows} & 1. Người dùng nhấn vào biểu tượng "Tin nhắn" trên màn hình trang chủ. \newline 2. Hệ thống hiển thị danh sách các tin nhắn. \newline 3. Người dùng có thể nhấn biểu tượng "Tìm kiếm" và nhập tên Người dùng khác/Nhóm để tìm tin nhắn cần xem $\rightarrow$ hệ thống gọi \textbf{UC-CHAT-02}. \newline 4. Người dùng có thể chọn Người dùng khác/nhóm với tin nhắn cần xem. \newline 5. Hệ thống chuyển vào hộp thoại nhắn tin.\\
    \hline
    \textbf{Alternative Flows} & \textbf{2.1} Không có tin nhắn hay thông báo $\rightarrow$ hệ thống hiện "Không có tin nhắn". \newline \textbf{3.1} Người dùng thoát khỏi danh sách tin nhắn mà không cần xem tin nhắn. \\
    \hline
    \textbf{Exception Flows} & \textbf{2.1.} Lỗi tải dữ liệu tin nhắn/thông báo, hệ thống in ra lỗi phù hợp.\\
    \hline
\end{longtable}

\newpage

\begin{longtable}{|p{0.2\linewidth}|p{0.75\linewidth}|}
    \caption{Bảng đặc tả Use-case: Nhắn tin (UC-CHAT-02)}. \\
    \hline
    \textbf{Use-case} & \textbf{UC-CHAT-03:} Nhắn tin. \\
    \hline
    \textbf{Actor} & Người dùng (Sinh viên/Tutor). \\
    \hline
    \textbf{Descriptions} & Người dùng nhắn tin trên hộp thoại tin nhắn. \\
    \hline
    \textbf{Include} & \textbf{UC-CHAT-01} Xem tin nhắn và thông báo. \newline \textbf{UC-CHAT-04} Tạo nhóm nhắn tin. \\
    \hline
    \textbf{Extend} & Không\\
    \hline
    \textbf{Precondition} & 1. Người dùng đã đăng nhập thành công. \newline 2. Người dùng đã chọn một hộp thoại nhắn tin từ danh sách tin nhắn xuất hiện ở \textbf{UC-CHAT-01}. \newline 3. Nếu nhắn tin trên nhóm thì nhóm phải được tạo từ Tutor bằng \textbf{UC-CHAT-04}.\\
    \hline
    \textbf{Postcondition} & Tin nhắn được gửi thành công. \\
    \hline
    \textbf{Trigger} & Không\\
    \hline
    \textbf{Normal Flows} & 1. Người dùng chọn một cuộc hội thoại cần nhắn tin trong danh sách tin nhắn xuất hiện ở \textbf{UC-CHAT-01}. \newline 2. Hệ thống hiển thị hộp thoại để nhắn tin. \newline 3. Người dùng nhập nội dung tin nhắn. \newline 4. Người dùng nhấn biểu tượng "Gửi" để gửi nội dung tin nhắn. \\
    \hline
    \textbf{Alternative Flows} & \textbf{2.1} Người dùng không nhắn tin, thoát khỏi hộp thoại. \newline \textbf{4.1} Người dùng không muốn gửi tin nhắn, xóa tất cả và thoát. \\
    \hline
    \textbf{Exception Flows} & \textbf{2.1.} Lỗi tải dữ liệu tin nhắn/thông báo, hệ thống in ra lỗi phù hợp.\\
    \hline
\end{longtable}
\newpage
\begin{longtable}{|p{0.2\linewidth}|p{0.75\linewidth}|}
    \caption{Bảng đặc tả Use-case: Tạo nhóm nhắn tin (UC-CHAT-04)}. \\
    \hline
    \textbf{Use-case} & \textbf{UC-CHAT-04:} Tạo nhóm.  \\
    \hline
    \textbf{Actor} & Tutor. \\
    \hline
    \textbf{Descriptions} & Tutor tạo một nhóm để nhắn tin cho lớp học của mình. \\
    \hline
    \textbf{Include} & Không \\
    \hline
    \textbf{Extend} & \textbf{UC-CHAT-05:} Tạo liên kết tham gia nhóm. \\
    \hline
    \textbf{Precondition} & 1. Tutor đã đăng nhập thành công. \\
    \hline
    \textbf{Postcondition} & Một nhóm nhắn tin mới được tạo. \\
    \hline
    \textbf{Trigger} & Biểu tượng "Tạo nhóm". \\
    \hline
    \textbf{Normal Flows} & 1. Tutor nhấn vào biểu tượng "Tạo nhóm" trên hộp danh sách tin nhắn. \newline 2. Hệ thống hiển thị hộp thoại "Tạo nhóm mới". \newline 3. Tutor đặt tên và chọn các sinh viên để thêm vào nhóm. \newline 4. Nếu số lượng sinh viên quá đông Tutor có thể nhấn vào biểu tượng "Chia sẻ link tham gia nhóm", hệ thống sẽ gọi \textbf{UC-CHAT-05} và tạo một liên kết tham gia nhóm và Tutor có thể chia sẻ trên bất cứ nền tảng nào. \newline 5. Tutor nhấn "Tạo nhóm" và nhóm mới sẽ được tạo.\\
    \hline
    \textbf{Alternative Flows} & 
    %\textbf{3.1} Tutor thực hiện tạo nhóm, thoát hộp thoại $\rightarrow$ quay về màn hình danh sách tin nhắn \newline
    \textbf{5.1} Tutor không tạo nhóm, thoát hộp thoại $\rightarrow$ quay về danh sách tin nhắn. \\
    \hline
    \textbf{Exception Flows} & \textbf{5.1.} Lỗi tạo nhóm $\rightarrow$ Hệ thống thông báo lỗi và quay về danh sách tin nhắn.\\
    \hline
\end{longtable}

\begin{longtable}{|p{0.2\linewidth}|p{0.75\linewidth}|}
    \caption{Bảng đặc tả Use-case: Gửi thông báo (UC-CHAT-06)}. \\
    \hline
    \textbf{Use-case} & \textbf{UC-CHAT-06:} Gửi thông báo.  \\
    \hline
    \textbf{Actor} & Nhà trường. \\
    \hline
    \textbf{Descriptions} & Nhà trường gửi thông báo chung cho sinh viên/Tutor. \\
    \hline
    \textbf{Include} & Không \\
    \hline
    \textbf{Extend} & Không \\
    \hline
    \textbf{Precondition} & Nhà trường đã đăng nhập thành công vào tài khoản. \\
    \hline
    \textbf{Postcondition} & Thông báo được gửi đi cho sinh viên/Tutor. \\
    \hline
    \textbf{Trigger} & Biểu tượng "Gửi thông báo". \\
    \hline
    \textbf{Normal Flows} & 1. Nhà trường nhấn vào biểu tượng "Gửi thông báo" trên màn hình chính. \newline 2. Hệ thống hiển thị hộp thoại gửi thông báo. \newline 3. Nhà trường nhập nội dung cần thông báo. \newline 4. Nhà trường nhấn biểu tượng "Gửi".\\
    \hline
    \textbf{Alternative Flows} & 
    %\textbf{3.1} Tutor thực hiện tạo nhóm, thoát hộp thoại $\rightarrow$ quay về màn hình danh sách tin nhắn \newline
    \textbf{4.1} Nhà trường không gửi thông báo $\rightarrow$ quay về màn hình chính. \\
    \hline
    \textbf{Exception Flows} & \textbf{4.1.} Lỗi gửi thông báo $\rightarrow$ Hệ thống thông báo lỗi và quay về màn hình chính.\\
    \hline
\end{longtable}