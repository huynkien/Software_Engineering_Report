\subsection{Đăng ký chương trình}

\begin{figure}[H]
    \centering
    \includegraphics[width=1\linewidth]{graphics/figures/registerprogram.png}
    \caption{Sơ đồ use-case cho chức năng Đăng ký chương trình học của sinh viên}
    \label{fig:registerprogram}
\end{figure}
\newpage
% \begin{longtable}{|p{0.2\linewidth}|p{0.8\linewidth}|}
%     \hline
%     \multicolumn{2}{|c|}{\textbf{Bảng đặc tả Use Case: Tìm tutor (UC1)}} \\
%     \hline
%     \endfirsthead
%     \hline
%     \endfoot
%     \endlastfoot
%     \textbf{Use-case} & Tìm tutor \\
%     \hline
%     \textbf{Actor} & Sinh viên \\
%     \hline
%     \textbf{Descriptions} & Sau khi chọn icon/item “Đăng ký” ở trang chính, hệ thống hiển thị các khóa học/tutor nổi bật (rating cao, hoạt động nhiều, liên quan đến các tìm kiếm trước đó (nếu có),...) dưới dạng card. Sinh viên có thể chọn 1 card để xem chi tiết. \\
%     \hline
%     \textbf{Trigger} & Sinh viên chọn icon/item ""Đăng ký"" ở trang chính \\
%     \hline
%     \textbf{Precondition} & 1. Sinh viên đã đăng nhập thành công. \newline 2. Sinh viên chọn "" Đăng ký “ ở trang chính. \\
%     \hline
%     \textbf{Postcondition} & 1. Danh sách các tutor/ khóa học nổi bật được hiển thị. \newline 2. Nếu Sinh viên bấm chọn 1 card, thì chuyển sang UC3 ""Chọn khóa học để xem”. \\
%     \hline
%     \textbf{Normal Flows} & 1. Hệ thống lấy dữ liệu từ tutor nổi bật. \newline 2. Hiển thị ở trang chính. \newline 3. Sinh viên chọn 1 khóa học/tutor để đăng ký $\rightarrow$ chuyển qua UC3. \\
%     \hline
%     \textbf{Alternative Flows} & A1- Sinh viên không chọn từ đề xuất mà dùng thanh tìm kiếm $\rightarrow$ UC2 ""Tìm kiếm”. \\
%     \hline
%     \textbf{Exception Flows} & E1 - Lỗi tải đề xuất: hiển thị thông báo lỗi và yêu cầu nhấn reload. \\
%     \hline
% \end{longtable}

% \begin{longtable}{|p{0.2\linewidth}|p{0.8\linewidth}|}
%     \hline
%     % \multicolumn{2}{|c|}{\textbf{UC1: Đề xuất tutor}} \\
%     % \hline
%     % \endfirsthead
%     % \hline
%     \textbf{Use-case} & UC-REG1: Đăng ký khóa học \\
%     \hline
%     \textbf{Actor} & Sinh viên \\
%     \hline
%     \textbf{Descriptions} & Sau khi chọn icon/item “Đăng ký” ở trang chính, hệ thống hiển thị các khóa học/tutor nổi bật (rating cao, hoạt động nhiều, liên quan đến các tìm kiếm trước đó (nếu có), ...) dưới dạng card. Sinh viên có thể chọn 1 card để xem chi tiết. \\
%     \hline
%     \textbf{Trigger} & Sinh viên chọn icon/item "Đăng ký" ở trang chính \\
%     \hline
%     \textbf{Precondition} & 1. Sinh viên đã đăng nhập thành công \newline 2. Sinh viên chọn " Đăng ký “ ở trang chính \\
%     \hline
%     \textbf{Postcondition} & 1. Danh sách các tutor/ khóa học nổi bật được hiển thị \newline 2. Nếu Sinh viên bấm chọn 1 card, thì chuyển sang \textbf{UC-REG3} "Chọn khóa học để xem” \\
%     \hline
%     \textbf{Normal Flows} & 1. Hệ thống lấy dữ liệu từ tutor nổi bật \newline 2. Hiển thị ở trang chính \newline 3. Sinh viên chọn 1 khóa học/tutor để đăng ký $\rightarrow$ chuyển qua \textbf{UC-REG3} \\
%     \hline
%     \textbf{Alternative Flows} & Sinh viên không chọn từ đề xuất mà dùng thanh tìm kiếm $\rightarrow$ \textbf{UC-REG2} " Tìm kiếm” \\
%     \hline
%     \textbf{Exception Flows} & Lỗi tải đề xuất : hiển thị thông báo lỗi và yêu cầu nhấn reload. \\
%     \hline
%     \end{longtable}

% \begin{longtable}{|p{0.2\linewidth}|p{0.8\linewidth}|}
%     \hline
%     % \multicolumn{2}{|c|}{\textbf{Bảng đặc tả Use Case: Tìm kiếm (UC2)}} \\
%     % \hline
%     % \endfirsthead
%     % \hline
%     % \endfoot
%     % \endlastfoot
%     \textbf{Use-case} & UC-REG2: Tìm kiếm \\
%     \hline
%     \textbf{Actor} & Sinh viên \\
%     \hline
%     \textbf{Descriptions} & Sinh viên nhập từ khóa vào thanh tìm kiếm và nhấn Enter. Hệ thống hiển thị trang gồm các khóa học liên quan, kèm filter để tùy chỉnh. \\
%     \hline
%     \textbf{Trigger} & Sinh viên nhấn Enter sau khi nhập thông tin trên thành tìm kiếm. \\
%     \hline
%     \textbf{Precondition} & 1. Sinh viên đã đăng nhập thành công. \newline 2. Sinh viên chọn "Đăng ký" ở trang chính. \\
%     \hline
%     \textbf{Postcondition} & Trang kết quả hiển thị danh các khóa học phù hợp và lưu trạng thái filter. \\
%     \hline
%     \textbf{Normal Flows} & 1. Sinh viên nhập nhu cầu trên thanh tìm kiếm và nhấn Enter. \newline 2. Hệ thống tìm khóa học/tutor phù hợp. \newline 3. Hiển thị danh sách khóa học, kèm filter. \newline 4. Sinh viên chọn 1 khóa học/tutor để đăng ký $\rightarrow$ Chuyển sang \textbf{UC-REG3}. \\
%     \hline
%     \textbf{Alternative Flows} & \textbf{Tại bước 2,} Không tìm thấy kết quả phù hợp: hiển thị thông báo. \\
%     \hline
%     \textbf{Exception Flows} & Không có \\
%     \hline
% \end{longtable}
% \newpage
% \begin{longtable}{|p{0.2\linewidth}|p{0.8\linewidth}|}
%     \hline
%     % \multicolumn{2}{|c|}{\textbf{Bảng đặc tả Use Case: Đăng ký khóa học (UC3)}} \\
%     % \hline
%     % \endfirsthead
%     % \hline
%     % \endfoot
%     % \endlastfoot
%     \textbf{Use-case} & UC-REG3: Chọn khóa học để xem \\
%     \hline
%     \textbf{Actor} & Sinh viên \\
%     \hline
%     \textbf{Descriptions} & Sinh viên chọn một khóa học trong danh sách (từ đề xuất hoặc kết quả tìm kiếm) để xem thông tin chi tiết và đăng ký. \\
%     \hline
%     \textbf{Trigger} & Sinh viên nhấn vào item/ card của khóa học. \\
%     \hline
%     \textbf{Precondition} & 1. Sinh viên đã đăng nhập thành công. \newline 2. Đã có danh sách khóa học từ \textbf{UC-REG1:} Đề xuất tutor hoặc \textbf{UC-REG2:} Tìm kiếm. \\
%     \hline
%     \textbf{Postcondition} & Đăng ký khóa học thành công. \\
%     \hline
%     \textbf{Normal Flows} & 1. Sinh viên chọn item/card khóa học từ đề xuất/kết quả tìm kiếm. \newline 2. Sinh viên có thể nhấn vào khóa học để xem thông tin về: \newline \qquad a. Mô tả khóa học \newline \qquad b. Hồ sơ tutor \newline \qquad c. Đánh giá về tutor \newline 3. Sinh viên nhấn đăng ký khóa học. \\
%     \hline
%     \textbf{Alternative Flows} & Sinh viên không đăng ký khóa học: giữ nguyên trang hiện tại. \\
%     \hline
%     \textbf{Exception Flows} & Khóa học vừa bị gỡ xuống: hệ thống báo lỗi và ở lại trang hiện tại. \\
%     \hline
% \end{longtable}
% % \newpage
% \begin{longtable}{|p{0.2\linewidth}|p{0.8\linewidth}|}
%     \hline
%     % \multicolumn{2}{|c|}{\textbf{Bảng đặc tả Use Case: Xem mô tả khóa học} \\
%     % \hline
%     % \endfirsthead
%     % \hline
%     % \endfoot
%     % \endlastfoot
%     \textbf{Use-case} & UC-REG4: Xem mô tả khóa học \\
%     \hline
%     \textbf{Actor} & Sinh viên \\
%     \hline
%     \textbf{Descriptions} & Hiển thị nội dung khóa học: mục tiêu, phạm vi. \\
%     \hline
%     \textbf{Trigger} & Tự động hiển thị trên trang chi tiết. \\
%     \hline
%     \textbf{Precondition} & Sinh viên bấm vào một khóa học. \\
%     \hline
%     \textbf{Postcondition} & Thông tin mô tả khóa học được hiển thị. \\
%     \hline
%     \textbf{Normal Flows} & Hệ thống tải và hiển thị thông tin khóa học. \\
%     \hline
%     \textbf{Alternative Flows} & Không có \\
%     \hline
%     \textbf{Exception Flows} & Lỗi tải thông tin. \\
%     \hline
% \end{longtable}
% \newpage
% \begin{longtable}{|p{0.2\linewidth}|p{0.8\linewidth}|}
%     \hline
%     % \multicolumn{2}{|c|}{\textbf{Bảng đặc tả Use Case: Xem hồ sơ tutor (UC3.2)}} \\
%     % \hline
%     % \endfirsthead
%     % \hline
%     % \endfoot
%     % \endlastfoot
%     \textbf{Use-case} & UC-REG5: Xem hồ sơ tutor \\
%     \hline
%     \textbf{Actor} & Sinh viên \\
%     \hline
%     \textbf{Descriptions} & Hiển thị hồ sơ tutor: chuyên ngành, lĩnh vực, kinh nghiệm,... \\
%     \hline
%     \textbf{Trigger} & Sinh viên chọn “Hồ sơ tutor". \\
%     \hline
%     \textbf{Precondition} & Sinh viên đang ở trang thông tin chi tiết về khóa học. \\
%     \hline
%     \textbf{Postcondition} & Thông tin hồ sơ tutor được hiển thị. \\
%     \hline
%     \textbf{Normal Flows} & Hệ thống tải và hiển thị hồ sơ tutor. \\
%     \hline
%     \textbf{Alternative Flows} & Không có \\
%     \hline
%     \textbf{Exception Flows} & Lỗi tải thông tin. \\
%     \hline
% \end{longtable}

% \begin{longtable}{|p{0.2\linewidth}|p{0.8\linewidth}|}
%     \hline
%     % \multicolumn{2}{|c|}{\textbf{Bảng đặc tả Use Case: Xem đánh giá về tutor (UC3.3)}} \\
%     % \hline
%     % \endfirsthead
%     % \hline
%     % \endfoot
%     % \endlastfoot
%     \textbf{Use-case} & UC-REG6: Xem đánh giá về tutor \\
%     \hline
%     \textbf{Actor} & Sinh viên \\
%     \hline
%     \textbf{Descriptions} & Hiển thị rating, nhận xét từ Sinh viên khác về tutor/khóa học. \\
%     \hline
%     \textbf{Trigger} & Sinh viên chọn "Xem đánh giá về tutor". \\
%     \hline
%     \textbf{Precondition} & Sinh viên đang ở trang thông tin chi tiết về khóa học. \\
%     \hline
%     \textbf{Postcondition} & Đánh giá về tutor được hiển thị. \\
%     \hline
%     \textbf{Normal Flows} & Hệ thống tải và hiển thị đánh giá về tutor. \\
%     \hline
%     \textbf{Alternative Flows} & Không có \\
%     \hline 
%     \textbf{Exception Flows} & Lỗi tải thông tin. \\
%     \hline
% \end{longtable}

\begin{longtable}{|p{0.2\linewidth}|p{0.75\linewidth}|}
    \caption{Bảng đặc tả Use-case: Đề xuất tutor (UC-REG-01)} \\
    \hline
    \textbf{Use-case} & \textbf{UC-REG-01:} Đề xuất tutor \\
    \hline
    \textbf{Actor} & Sinh viên \\
    \hline
    \textbf{Descriptions} & Sau khi chọn "Đăng ký" ở trang chính, hệ thống hiển thị các khóa học/tutor nổi bật (rating cao, hoạt động nhiều, liên quan đến các tìm kiếm trước đó (nếu có), ...) dưới dạng card. Sinh viên có thể chọn 1 card để xem chi tiết. \\
    \hline
    \textbf{Include} & Không \\
    \hline
    \textbf{Extend} & \textbf{UC-REG-03:} Đăng ký khóa học \\
    \hline
    \textbf{Precondition} & 1. Sinh viên đã đăng nhập thành công và chọn " Đăng ký “ ở trang chính \\
    \hline
    \textbf{Postcondition} & 1. Danh sách các tutor/khóa học nổi bật được hiển thị \newline 2. Nếu sinh viên bấm chọn 1 card, thì chuyển sang \textbf{UC-REG-03} \\
    \hline
    \textbf{Trigger} & Sinh viên chọn icon "Đăng ký" ở trang chính \\
    \hline
    \textbf{Normal Flows} & 1. Hệ thống lấy dữ liệu từ tutor nổi bật \newline 2. Hiển thị ở trang chính \newline 3. Sinh viên chọn 1 khóa học/tutor để đăng ký $\rightarrow$ chuyển qua \textbf{UC-REG-03} \\
    \hline
    \textbf{Alternative Flows} & Không \\
    \hline
    \textbf{Exception Flows} & \textbf{2.1.} Lỗi tải đề xuất: hiển thị thông báo lỗi và yêu cầu nhấn tải lại. \\
    \hline
\end{longtable}
\begin{longtable}{|p{0.2\linewidth}|p{0.75\linewidth}|}
    \caption{Bảng đặc tả Use-case: Tìm kiếm (UC-REG-02)} \\
    \hline
    \textbf{Use-case} & \textbf{UC-REG-02:} Tìm kiếm \\
    \hline
    \textbf{Actor} & Sinh viên \\
    \hline
    \textbf{Descriptions} & Sinh viên nhập từ khóa vào thanh tìm kiếm và nhấn Enter. Hệ thống hiển thị trang gồm các khóa học liên quan, kèm filter để tùy chỉnh. \\
    \hline
    \textbf{Include} & \textbf{UC-REG-03:} Đăng ký khóa học \\
    \hline
    \textbf{Extend} & Không \\
    \hline
    \textbf{Precondition} & 1. Sinh viên đã đăng nhập thành công \newline 2. Sinh viên chọn "Đăng ký“ ở trang chính \\
    \hline
    \textbf{Postcondition} & Trang kết quả hiển thị danh các card khóa học phù hợp và lưu trạng thái filter \\
    \hline
    \textbf{Trigger} & Sinh viên nhấn Enter sau khi nhập thông tin trên thanh tìm kiếm \\
    \hline
    \textbf{Normal Flows} & 1. Sinh viên nhập nhu cầu trên thanh tìm kiếm và nhấn Enter \newline 2. Hệ thống tìm khóa học/tutor phù hợp \newline 3. Hiển thị danh sách khóa học, kèm filter \newline 4. Sinh viên chọn 1 khóa học/tutor để đăng ký $\rightarrow$ Chuyển sang \textbf{UC-REG-03} \\
    \hline
    \textbf{Alternative Flows} & \textbf{2.1.} Không tìm thấy kết quả phù hợp: hiển thị thông báo \\
    \hline
    \textbf{Exception Flows} & Không\\
    \hline
\end{longtable}

\begin{longtable}{|p{0.2\linewidth}|p{0.75\linewidth}|}
    \caption{Bảng đặc tả Use-case: Đăng ký khóa học (UC-REG-03)} \\
    \hline
    \textbf{Use-case} & \textbf{UC-REG-03:} Đăng ký khóa học \\
    \hline
    \textbf{Actor} & Sinh viên \\
    \hline
    \textbf{Descriptions} & Sinh viên chọn một khóa học trong danh sách (từ đề xuất hoặc kết quả tìm kiếm) để xem thông tin chi tiết và đăng ký \\
    \hline
    \textbf{Include} & Không \\    
    \hline
    \textbf{Extend} & \textbf{UC-REG-04:} Xem mô tả khóa học \newline \textbf{UC-REG-05:} Xem hồ sơ tutor \newline \textbf{UC-REG-06:} Xem đánh giá về tutor \\
    \hline
    \textbf{Precondition} & 1. Sinh viên đã đăng nhập thành công \newline 2. Đã có danh sách khóa học từ \textbf{UC-REG-01:} Đề xuất tutor hoặc \textbf{UC-REG-02:} Tìm kiếm \\
    \hline
    \textbf{Postcondition} & Đăng ký khóa học thành công \\
    \hline
    \textbf{Trigger} & Sinh viên nhấn vào item/card của khóa học \\
    \hline
    \textbf{Normal Flows} & 1. Sinh viên chọn item/card khóa học từ đề xuất/kết quả tìm kiếm \newline 2. Sinh viên có thể nhấn vào khóa học để xem thông tin về: \newline \qquad a. \textbf{UC-REG-04:} Xem mô tả khóa học \newline \qquad b.\textbf{ UC-REG-05:} Xem hồ sơ tutor \newline \qquad c. \textbf{UC-REG-06:} Xem đánh giá về tutor \newline 3. Sinh viên nhấn đăng ký khóa học \\
    \hline
    \textbf{Alternative Flows} & \textbf{3.1.} Sinh viên không đăng ký khóa học: giữ nguyên page hiện tại. \\
    \hline
    \textbf{Exception Flows} & \textbf{3.1.} Khóa học vừa bị gỡ xuống: hệ thống báo lỗi và ở lại trang hiện tại \\
    \hline
\end{longtable}

\begin{longtable}{|p{0.2\linewidth}|p{0.75\linewidth}|}
    \caption{Bảng đặc tả Use-case: Xem mô tả khóa học (UC-REG-04)} \\
    \hline
    \textbf{Use-case} & \textbf{UC-REG-04:} Xem mô tả khóa học \\
    \hline
    \textbf{Actor} & Sinh viên \\
    \hline
    \textbf{Descriptions} & Hiển thị nội dung khóa học: mục tiêu, phạm vi,... \\
    \hline
    \textbf{Inlcude} & Không \\
    \hline
    \textbf{Extend} & Không \\
    \hline
    \textbf{Precondition} & Sinh viên bấm vào một khóa học \\
    \hline
    \textbf{Postcondition} & Thông tin mô tả khóa học được hiển thị \\
    \hline
    \textbf{Trigger} & Tự động hiển thị trên trang chi tiết \\
    \hline
    \textbf{Normal Flows} & Hệ thống tải và hiển thị thông tin khóa học \\
    \hline
    \textbf{Alternative Flows} & Không \\
    \hline
    \textbf{Exception Flows} & Lỗi tải thông tin \\
    \hline
\end{longtable}

\begin{longtable}{|p{0.2\linewidth}|p{0.75\linewidth}|}
    \caption{Bảng đặc tả Use-case: Xem hồ sơ tutor (UC-REG-05)} \\
    \hline
    \textbf{Use-case} & \textbf{UC-REG-05:} Xem hồ sơ tutor \\
    \hline
    \textbf{Actor} & Sinh viên \\
    \hline
    \textbf{Descriptions} & Hiển thị hồ sơ tutor: chuyên ngành, lĩnh vực, kinh nghiệm,... \\
    \hline
    \textbf{Inlcude} & Không \\
    \hline
    \textbf{Extend} & Không \\
    \hline
    \textbf{Precondition} & Sinh viên đang ở trang thông tin chi tiết về khóa học \\
    \hline
    \textbf{Postcondition} & Thông tin hồ sơ tutor được hiển thị \\
    \hline
    \textbf{Trigger} & Sinh viên chọn “Hồ sơ tutor” \\
    \hline
    \textbf{Normal Flows} & Hệ thống tải và hiển thị hồ sơ tutor \\
    \hline
    \textbf{Alternative Flows} & Không \\
    \hline
    \textbf{Exception Flows} & Lỗi tải thông tin \\
    \hline
\end{longtable}

\begin{longtable}{|p{0.2\linewidth}|p{0.75\linewidth}|}
    \caption{Bảng đặc tả Use-case: Xem đánh giá về tutor (UC-REG-06)} \\
    \hline
    \textbf{Use-case} & \textbf{UC-REG-06:} Xem đánh giá về tutor \\
    \hline
    \textbf{Actor} & Sinh viên \\
    \hline
    \textbf{Descriptions} & Hiển thị rating, nhận xét từ sinh viên khác về tutor/khóa học \\
    \hline
    \textbf{Inlcude} & Không \\
    \hline
    \textbf{Extend} & Không \\
    \hline
    \textbf{Precondition} & Sinh viên đang ở trang thông tin chi tiết về khóa học \\
    \hline
    \textbf{Postcondition} & Đánh giá về tutor được hiển thị \\
    \hline
    \textbf{Trigger} & Sinh viên chọn “Xem đánh giá về tutor” \\
    \hline
    \textbf{Normal Flows} & Hệ thống tải và hiển thị đánh giá về tutor \\
    \hline
    \textbf{Alternative Flows} & Không \\
    \hline
    \textbf{Exception Flows} & Lỗi tải thông tin \\
    \hline
\end{longtable}

\newpage