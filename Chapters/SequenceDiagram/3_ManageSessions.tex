\subsection{Tạo chương trình học}

\begin{figure}[H]
    \centering
    \includegraphics[width=0.95\linewidth]{graphics/SequenceDiagram/ManageSessionSequence/sequen_diagram_tao_lop.png}
    \caption{Sơ đồ tuần tự cho use-case "Tạo lớp học"}
    \label{fig:taolophoc}
\end{figure}
Sequence diagram UC-CLASS-02 mô tả quá trình tutor yêu cầu tạo một lớp học mới, 
hệ thống hiển thị form nhập liệu, kiểm tra dữ liệu và lưu thông tin lớp vào cơ sở dữ liệu. 
Quy trình có sự tương tác giữa năm thành phần: \textit{Tutor}, \textit{Controller}, 
\textit{Service}, \textit{Repository} và \textit{Database}. Chi tiết như sau:

1. Hiển thị form tạo lớp

\begin{itemize}
    \item Tutor gửi yêu cầu \texttt{requestClassCreationForm()} đến Controller để hiển thị form tạo lớp.
    \item Controller gọi phương thức \texttt{getClassFormTemplate()} đến Service.
    \item Service truy cập Repository để lấy mẫu form (nếu cần).
    \item Service gửi lại \texttt{formTemplate} cho Controller.
    \item Controller hiển thị form tạo lớp cho Tutor bằng \texttt{displayClassForm(formTemplate)}.
\end{itemize}

2. Điền thông tin và lưu

{2.1 Tutor gửi thông tin lớp}
\begin{itemize}
    \item Tutor nhập thông tin lớp học và gửi dữ liệu đến Controller thông qua 
    \texttt{submitClassInfo(classData)}.
    \item Controller chuyển dữ liệu này cho Service bằng phương thức 
    \texttt{validateClassInfo(classData)}.
\end{itemize}

2.2 Kiểm tra tính hợp lệ

\begin{itemize}
    \item Service gọi \texttt{checkClassData(classData)} đến Repository để kiểm tra dữ liệu.
    \item Repository truy vấn cơ sở dữ liệu qua \texttt{queryValidation(classData)}.
    \item Database trả về kết quả kiểm tra (\texttt{validationStatus}).
    \item Repository gửi \texttt{validationStatus} (valid/invalid) về Service.
\end{itemize}

2.3 Xử lý theo điều kiện
Trường hợp 1: Thông tin hợp lệ (\texttt{validationStatus = valid})

\begin{itemize}
    \item Service gọi \texttt{insertClass(classData)} đến Repository để lưu lớp học.
    \item Repository thực hiện lưu dữ liệu lớp bằng \texttt{insertClassRecord(classData)}.
    \item Database trả về \texttt{success(true)} cho Repository.

    \item Repository tiếp tục thực hiện \texttt{insertScheduleRecord(classId)} 
    để thêm lớp học vào lịch giảng dạy.
    \item Database phản hồi \texttt{success(true)}.

    \item Repository trả kết quả thành công cho Service.
    \item Service phản hồi Controller bằng \texttt{success("Class created successfully")}.
    \item Controller thông báo kết quả cho Tutor qua \texttt{notifySuccess("Tạo lớp thành công")}.
\end{itemize}

Trường hợp 2: Thông tin không hợp lệ (\texttt{validationStatus = invalid})

\begin{itemize}
    \item Service trả về lỗi cho Controller bằng \texttt{error("Invalid data")}.
    \item Controller thông báo lỗi cho Tutor bằng \texttt{notifyError("Sai thông tin, nhập lại")}.
\end{itemize}

3. Kết thúc

Quy trình kết thúc khi hệ thống phản hồi kết quả cho Tutor, bao gồm:
tạo lớp thành công hoặc thông báo lỗi yêu cầu nhập lại thông tin.




\begin{figure}[H]
    \centering
    \includegraphics[width=0.95\linewidth]{graphics/SequenceDiagram/ManageSessionSequence/sequen_diagram_quan_l_y_lop.png}
    \caption{Sơ đồ tuần tự cho use-case "Quản lý lớp học"}
    \label{fig:quanlylophoc}
\end{figure}
Sequence diagram UC-CLASS-03 minh hoạ quy trình tutor quản lý lớp học và lịch dạy thông qua các hành động: 
hiển thị danh sách lớp, xem chi tiết lớp, chỉnh sửa–huỷ lớp và cập nhật thông tin lớp.

1. Hiển thị danh sách lớp
\begin{itemize}
    \item Tutor gửi yêu cầu \texttt{requestClassList()} đến Controller.
    \item Controller gọi \texttt{getClassList()} ở Service.
    \item Service tiếp tục gọi \texttt{fetchClassList()} đến Repository.
    \item Repository truy vấn CSDL bằng \texttt{queryClassList()}.
    \item Danh sách lớp \texttt{classList} được trả về theo chiều ngược lại qua Repository → Service → Controller.
    \item Controller hiển thị danh sách lớp cho Tutor: \texttt{displayClassList(classList)}.
\end{itemize}

2. Xem chi tiết lớp (tùy chọn)
\begin{itemize}
    \item Tutor chọn một lớp để xem chi tiết và gọi \texttt{requestClassDetails(classId)}.
    \item Controller gọi \texttt{getClassDetailsById(classId)} ở Service.
    \item Service gọi tiếp \texttt{queryClassDetails(classId)} tại Repository.
    \item Repository truy vấn database và trả về thông tin chi tiết lớp \texttt{classDetails}.
    \item Thông tin được chuyển ngược lại lên Controller.
    \item Controller hiển thị chi tiết lớp: \texttt{displayClassDetails(classDetails)}.
\end{itemize}

3. Dùng chức năng huỷ / chỉnh sửa (tùy chọn)
\begin{itemize}
    \item Tutor yêu cầu chỉnh sửa/hủy thông tin lớp qua \texttt{requestClassStatusUpdate(classId)}.
    \item Controller gọi \texttt{fetchClassStatus(classId)} tại Service.
    \item Service gọi Repository với \texttt{queryCurrentClassState(classId)}.
    \item Repository kiểm tra database và trả về trạng thái hiện tại \texttt{currentStatus}.
    \item Trạng thái được truyền lên lại Controller.
    \item Controller hiển thị màn hình chỉnh sửa: \texttt{displayClassStateForEdit(currentStatus)}.
\end{itemize}

4. Sửa thông tin và lưu
\begin{itemize}
    \item Tutor gửi yêu cầu lưu thay đổi: \texttt{updateClassInfo(classData)} đến Controller.
    \item Controller gọi Service: \texttt{validateAndSaveClass(classData)}.
    \item Service thực hiện kiểm tra trùng lịch bằng \texttt{checkTimeConflict(classData)}.
    \item Repository thực hiện truy vấn \texttt{queryOverlappingSchedule(classData)}.
    \item Nếu trả về \texttt{conflictStatus = false} nghĩa là không trùng lịch:
    \begin{itemize}
        \item Service gọi Repository lưu dữ liệu: \texttt{updateClassRecord(classData)}.
        \item Repository trả về \texttt{success(true)}.
        \item Controller thông báo thành công: \texttt{notify\_UpdateSuccess}.
    \end{itemize}
    \item Nếu có trùng lịch:
    \begin{itemize}
        \item Service trả về lỗi \texttt{error("Schedule conflict")}.
        \item Controller hiển thị thông báo lỗi: \texttt{notify\_ConflictError}.
    \end{itemize}
\end{itemize}




\begin{figure}[H]
    \centering
    \includegraphics[width=0.95\linewidth]{graphics/SequenceDiagram/ManageSessionSequence/sequen_diagram_lich_ranh.png}
    \caption{Sơ đồ tuần tự cho use-case "Quản lý lịch rảnh"}
    \label{fig:quanlylichranh}
\end{figure}
Sequence diagram UC-CLASS-01 mô tả quá trình tutor xem danh sách lịch rảnh và thêm 
một lịch rảnh mới. Quy trình bao gồm sự tương tác giữa năm thành phần: 
\textit{Tutor}, \textit{Controller}, \textit{Service}, \textit{Repository} và \textit{Database}. 
Chi tiết như sau:

1. Hiển thị danh sách lịch rảnh

\begin{itemize}
    \item Tutor gửi yêu cầu \texttt{requestExistingSchedules()} đến Controller để lấy danh sách lịch rảnh.
    \item Controller gọi \texttt{getExistingSchedules()} đến Service.
    \item Service tiếp tục gọi \texttt{fetchExistingSchedules()} đến Repository.
    \item Repository truy vấn cơ sở dữ liệu thông qua \texttt{queryAvailableSchedules()}.
    \item Database trả về danh sách lịch rảnh (\textit{scheduleList}) cho Repository.
    \item Repository gửi \textit{scheduleList} về Service.
    \item Service trả danh sách lịch rảnh lại cho Controller.
    \item Controller hiển thị thông tin cho Tutor bằng lời gọi \texttt{displayAvailableSchedules(scheduleList)}.
\end{itemize}

2. Thêm lịch rảnh mới

2.1 Tutor gửi yêu cầu
\begin{itemize}
    \item Tutor gửi yêu cầu thêm lịch rảnh mới thông qua \texttt{submitNewSchedule(time, type)} đến Controller.
    \item Controller chuyển yêu cầu đến Service qua phương thức \texttt{validateAndSaveSchedule(time, type)}.
\end{itemize}

2.2 Kiểm tra trùng lịch
\begin{itemize}
    \item Service gọi \texttt{checkDuplicateSchedule(time)} đến Repository để kiểm tra lịch trùng.
    \item Repository truy vấn Database bằng \texttt{queryDuplicateSchedule(time)}.
    \item Database trả về trạng thái trùng lịch (\texttt{duplicateStatus}).
    \item Repository gửi \texttt{duplicateStatus} (true/false) về Service.
\end{itemize}
2.3 Xử lý theo điều kiện

Trường hợp 1: Không trùng lịch (\texttt{duplicateStatus = false})
\begin{itemize}
    \item Service gọi \texttt{insertNewSchedule(time, type)} đến Repository để lưu lịch.
    \item Repository thực hiện thêm bản ghi bằng \texttt{insertRecord(time, type)}.
    \item Database trả về kết quả thành công.
    \item Repository trả về \texttt{success(true)} cho Service.
    \item Service phản hồi Controller bằng \texttt{success("Save successful")}.
    \item Controller thông báo kết quả thành công cho Tutor qua \texttt{notify SaveSuccess}.
\end{itemize}

Trường hợp 2: Trùng lịch (\texttt{duplicateStatus = true}
\begin{itemize}
    \item Service trả về lỗi \texttt{return error("Duplicate schedule")} cho Controller.
    \item Controller thông báo cho Tutor thông qua \texttt{notify DuplicateError}.
\end{itemize}

3. Kết thúc
Quy trình kết thúc khi hệ thống phản hồi kết quả cho Tutor, bao gồm:
thêm lịch thành công hoặc thông báo lỗi do trùng lịch.


\newpage