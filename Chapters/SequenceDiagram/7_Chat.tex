\subsection{Thông báo và nhắn tin}

\begin{figure}[H]
    \centering
    \includegraphics[width=0.9\linewidth]{graphics/SequenceDiagram/ChatSequence/XemThongBaovaTinNhan1.png}
    \caption{Sơ đồ tuần tự cho use-case "Xem tin nhắn và thông báo"}
    \label{fig:xemtinnhan}
\end{figure}
\subsection*{Mô tả: Xem thông báo và tin nhắn}
\begin{enumerate}[leftmargin=*, label={\alph*.}]
    \item Người dùng lấy danh sách các cuộc trò chuyện gần đây qua \texttt{GET /conversations/recent}: Controller gọi TextingService, truy vấn ConversationRepository, trả về các hội thoại gần nhất.
    \item Người dùng thực hiện tìm kiếm hội thoại với từ khóa qua \texttt{GET /conversations/search?keyword} Controller gọi SearchConversationService, truy ConversationRepository với điều kiện keyword, trả về kết quả phù hợp (hoặc phản hồi rỗng).
    \item Người dùng xem lịch sử tin nhắn cuộc trò chuyện qua \texttt{GET /conversations/\{id\}/messages}: Controller gọi TextingService, repository truy vấn message theo conversationId, trả về lịch sử tin nhắn.
\end{enumerate}

\newpage
\begin{figure}[H]
    \centering
    \includegraphics[width=1\linewidth]{graphics/SequenceDiagram/ChatSequence/TaoNhom1.png}
    \caption{Sơ đồ tuần tự cho use-case "Tạo nhóm nhắn tin"}
    \label{fig:taonhom}
\end{figure}
\newpage
\subsection*{Mô tả: Tạo nhóm nhắn tin}
\begin{enumerate}[leftmargin=*, label={\alph*.}]
    \item Tutor gửi yêu cầu mở giao diện tạo nhóm (\texttt{GET /group/create}) và nhận phản hồi xác nhận.
    \item Tutor điền thông tin tên nhóm và danh sách thành viên rồi gửi lên (\texttt{POST /group}), Controller tiếp nhận chuyển cho CreateGroupService.
    \item Dịch vụ này gọi GroupRepository, lưu group mới vào DB (ghi nhận groupName, thành viên, ngày tạo).
    \item Kết quả lưu thành công được trả về (201 Created).
    \item Tutor có thể tạo liên kết tham gia bằng cách gửi (\texttt{POST /group/\{groupId\}/joinlink}), dịch vụ sẽ sinh token link, lưu vào DB.
    \item Nếu bỏ qua, hệ thống ghi nhận thao tác và không tạo link.
    \item Cuối cùng, tutor truy xuất giao diện chat nhóm vừa tạo.
\end{enumerate}


\newpage
\begin{figure}[H]
    \centering
    \includegraphics[width=1\linewidth]{graphics/SequenceDiagram/ChatSequence/NhanTin1.png}
    \caption{Sơ đồ tuần tự cho use-case "Nhắn tin"}
    \label{fig:nhantin}
\end{figure}

\subsection*{Mô tả: Nhắn tin}
\begin{enumerate}[leftmargin=*, label={\alph*.}]
    \item Người dùng gửi yêu cầu POST nhắn tin mới, gồm nội dung và conversationId.
    \item Controller tiếp nhận chuyển đến TextingService, dịch vụ này khởi tạo message object và lưu thông qua MessageRepository vào DB.
    \item Sau khi lưu thành công, dịch vụ tiếp tục gọi NotificationService để gửi thông báo tới các thành viên hội thoại vừa nhận tin nhắn.
    \item Luồng kết thúc khi hệ thống trả về phản hồi xác nhận gửi thành công cho người dùng.
\end{enumerate}


\newpage
\begin{figure}[H]
    \centering
    \includegraphics[width=1\linewidth]{graphics/SequenceDiagram/ChatSequence/GuiThongBao1.png}
    \caption{Sơ đồ tuần tự cho use-case "Gửi thông báo"}
    \label{fig:guiThongBao}
\end{figure}
\subsection*{Mô tả: Gửi thông báo}
\begin{enumerate}[leftmargin=*, label={\alph*.}]
    \item Nhà trường mở giao diện tạo thông báo (\texttt{GET /announcement/create}), nhận lại form nhập liệu.
    \item Gửi thông báo mới (\texttt{POST /announcement}) với tiêu đề, nội dung, loại đối tượng nhận: Controller gửi tới AnnouncementService.
    \item AnnouncementService xử lý, lưu vào AnnouncementRepository, ghi nhận các thông tin về thông báo.
    \item Thông tin thông báo mới được chuyển tiếp đến NotificationService để phân phối đến danh sách người nhận thỏa mãn điều kiện (recipientType).
    \item Hệ thống trả về các trạng thái: gửi thành công (200 OK); lỗi xác thực (403 Forbidden); không có người nhận phù hợp (404 Not Found).
\end{enumerate}
\newpage