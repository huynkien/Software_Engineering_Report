\subsection{Thiết lập lịch trình cho sinh viên}

\begin{figure}[H]
    \centering
    \includegraphics[width=1\linewidth]{graphics/SequenceDiagram/SchedulingSequence/DoiLich.jpg}
    \caption{Sơ đồ tuần tự cho use-case "Đổi lịch học" cho sinh viên"}
    \label{fig:doilichchosinhvien}
\end{figure}

\subsection*{Mô tả sơ đồ tuần tự: Đổi lịch học}

\paragraph{Bước 1: Chọn chức năng đổi lịch.}
\begin{itemize}[leftmargin=1.5cm]
    \item Sinh viên chọn chức năng đổi lịch trên giao diện, tương ứng với thông điệp \texttt{requestChangeSchedule()} gửi đến \texttt{Controller}.
    \item \texttt{Controller} gọi phương thức \texttt{getAvailableClasses(userID)} của \texttt{Service} để lấy danh sách lớp mà sinh viên đã đăng ký.
    \item \texttt{Service} tiếp tục gọi \texttt{findRegisteredClasses(userID)} trên \texttt{Repo}. 
    \item \texttt{Repo} truy vấn \texttt{db.registrations.find(\{ userID: userID \})} để lấy danh sách đăng ký và trả về \texttt{List<Registration>} cho \texttt{Service}.
    \item \texttt{Service} chuyển danh sách này thành \texttt{availableClasses} và trả về cho \texttt{Controller}.
    \item \texttt{Controller} gửi lại kết quả cho sinh viên thông qua thông điệp 

    \texttt{responseAvailableClasses(availableClasses)} và giao diện hiển thị danh sách lớp có thể đổi.
\end{itemize}

\paragraph{Bước 2: Chọn lớp cần đổi và gửi yêu cầu đổi lịch.}
\begin{itemize}[leftmargin=1.5cm]
    \item Sau khi chọn lớp và ca học mới, sinh viên gửi yêu cầu \texttt{submitChangeRequest(classID, newSchedule)} tới \texttt{Controller}.
    \item \texttt{Controller} gọi \texttt{changeSchedule(classID, newSchedule)} trên \texttt{Service}.
    \item \texttt{Service} trước hết gọi \texttt{checkScheduleConflict(classID, newSchedule)}. Bên trong, \texttt{Repo} thực hiện truy vấn \texttt{db.schedules.find(\{ classID: classID, schedule: newSchedule \})} để kiểm tra trùng lịch và trả về \texttt{conflictCount}.
    \item Biểu đồ sử dụng khối \texttt{alt} để tách hai trường hợp:
    \begin{itemize}
        \item \textbf{Trường hợp lịch học bị trùng:} nếu \texttt{conflictCount > 0}, 
        \texttt{Service} trả về thông điệp \texttt{returnError("Schedule conflict detected")} cho \texttt{Controller}. 
        Controller hiển thị thông báo lỗi cho sinh viên thông qua \texttt{showError("Cannot change schedule due to conflict")}.
        \item \textbf{Trường hợp lịch học hợp lệ:} nếu không có xung đột, 
        \texttt{Service} gọi hàm 
        
        \texttt{updateClassSchedule(classID, newSchedule)} trên \texttt{Repo}. 
        Repo cập nhật bản ghi tương ứng trong cơ sở dữ liệu 
        % (ví dụ \texttt{db.classes.updateOne(\{\_id: classID\}, \{\$set: \{ schedule: newSchedule\}\})}) 
        và trả về kết quả \texttt{success(true)}.
        \item Sau khi cập nhật thành công, \texttt{Service} gửi lại thông điệp \\
        \texttt{responseSuccess("Schedule changed successfully")} cho \texttt{Controller}.
        \item \texttt{Controller} hiển thị thông báo thành công cho sinh viên thông qua 
        \texttt{showMessage("Schedule change completed successfully")}.
    \end{itemize}
\end{itemize}
\newpage
\begin{figure}[H]
    \centering
    \includegraphics[width=0.93\linewidth]{graphics/SequenceDiagram/SchedulingSequence/HuyLich.jpg}
    \caption{Sơ đồ tuần tự cho use-case "Hủy lịch học" cho sinh viên"}
    \label{fig:huylichhocchosinhvien}
\end{figure}

\subsection*{Mô tả sơ đồ tuần tự: Hủy lịch học}

\paragraph{Bước 1: Chọn chức năng hủy lịch.}
\begin{itemize}
    \item Sinh viên chọn chức năng hủy lịch trên giao diện, tương ứng với thông điệp \texttt{cancelSchedule(request)} gửi đến \texttt{Controller}.
    \item \texttt{Controller} gọi \texttt{findRegisterClasses(userID)} trên \texttt{Service} để lấy danh sách các lớp mà sinh viên đã đăng ký.
    \item \texttt{Service} tiếp tục gọi \texttt{queryRequestData(userID)} trên \texttt{Repo}.
    \item \texttt{Repo} truy vấn \texttt{db.registrations.find(\{ userID: userID \})} trên cơ sở dữ liệu và nhận về \texttt{List<Registration>}.
    \item Danh sách này được trả ngược lại cho \texttt{Service}, sau đó \texttt{Service} gửi thông điệp \texttt{showRegisteredClasses} cho \texttt{Controller}.
    \item \texttt{Controller} hiển thị danh sách lớp cho sinh viên thông qua lời gọi \texttt{hiểnThiDanhSáchLớp()} ở phía giao diện.
\end{itemize}

\paragraph{Bước 2: Chọn lớp cần hủy và xử lý yêu cầu.}
\begin{itemize}
    \item Sinh viên chọn một lớp trong danh sách, gửi thông điệp \\
    \texttt{selectClassToCancel(classID)} tới \texttt{Controller}.
    \item \texttt{Controller} gọi \texttt{validateCancel(classID, userID)} trên \texttt{Service} để kiểm tra xem lớp này có được phép hủy hay không.
    \item \texttt{Service} gọi tiếp \texttt{checkCancelCondition(classID, userID)} trên \texttt{Repo}.
    \item \texttt{Repo} truy vấn \texttt{db.conditions.find(\{ classID: classID, userID: userID \})} và trả về \texttt{conditionStatus} phản ánh kết quả kiểm tra điều kiện hủy.
\end{itemize}

Sau khi nhận được \texttt{conditionStatus}, biểu đồ sử dụng khối \texttt{alt} để mô tả hai nhánh xử lý:

\begin{itemize}
    \item \textbf{Nhánh ``Hủy không hợp lệ'':}
    \begin{itemize}
        \item Nếu điều kiện không thỏa (ví dụ đã quá hạn hủy), \texttt{Service} gửi thông điệp \texttt{Error: NotAllowed} về cho \texttt{Controller}.
        \item \texttt{Controller} hiển thị thông báo lỗi cho sinh viên qua lời gọi \texttt{showError("Cannot cancel this schedule")}.
    \end{itemize}

    \item \textbf{Nhánh ``Hủy hợp lệ'':}
    \begin{itemize}
        \item Nếu việc hủy được phép, \texttt{Service} gọi \\
        \texttt{deleteSchedule(classID)} trên \texttt{Repo}.
        \item \texttt{Repo} thực hiện thao tác xóa trong cơ sở dữ liệu (ví dụ \\
        \texttt{db.classes.deleteOne(\{\_id: classID\})}) và nhận phản hồi \texttt{Delete success}.
        \item Kết quả \texttt{success} được gửi ngược lại cho \texttt{Service}, sau đó \texttt{Service} trả về \texttt{ScheduleCancelResponse(success)} cho \texttt{Controller}.
        \item Cuối cùng, \texttt{Controller} hiển thị thông báo \texttt{showMessage("Schedule canceled successfully")} để xác nhận hủy lịch thành công cho sinh viên.
    \end{itemize}
\end{itemize}

\begin{figure}[H]
    \centering
    \includegraphics[width=1\linewidth]{graphics/SequenceDiagram/SchedulingSequence/XemLaiLich.jpg}
    \caption{Sơ đồ tuần tự cho use-case "Xem lại lịch học" cho sinh viên"}
    \label{fig:xemlailichchosinhvien}
\end{figure}

\subsection*{Mô tả sơ đồ tuần tự: Xem lại lịch học}

\paragraph{Bước 1: Xem lịch học tuần hiện tại.}
\begin{itemize}
    \item Sinh viên yêu cầu xem lịch hiện tại thông qua thông điệp 
    \texttt{getSchedule(userID, currentWeek)} gửi đến \texttt{Controller}.
    \item \texttt{Controller} chuyển tiếp yêu cầu sang \texttt{Service} bằng lời gọi\\
    \texttt{getScheduleData(userID, currentWeek)}.
    \item \texttt{Service} gọi \texttt{querySchedule(userID, currentWeek)} trên \texttt{Repo} để truy vấn dữ liệu lịch học.
    \item \texttt{Repo} truy vấn cơ sở dữ liệu và nhận về \texttt{List<Schedule>}.
    \item Danh sách này lần lượt được trả ngược về \texttt{Service} và \texttt{Controller} thông qua thông điệp \texttt{ScheduleList}.
    \item Cuối cùng, \texttt{Controller} hiển thị lịch học tuần hiện tại cho sinh viên thông qua lời gọi \\ \texttt{displayCurrentWeekSchedule()}.
\end{itemize}
\newpage
\paragraph{Bước 2: Lọc lịch học (tùy chọn).}
\begin{itemize}
    \item Khi sinh viên muốn xem lịch theo tuần khác hoặc lọc theo môn học, 
    sinh viên gửi yêu cầu \texttt{getFilteredSchedule(userID, criteria)} tới \texttt{Controller}. 
    (Khối \texttt{opt} trong sơ đồ cho biết bước này là tuỳ chọn.)
    \item \texttt{Controller} gọi \texttt{getFilteredData(userID, criteria)} trên \texttt{Service}.
    \item \texttt{Service} gọi \texttt{queryFilteredSchedule(userID, criteria)} trên \texttt{Repo}.
    \item \texttt{Repo} truy vấn cơ sở dữ liệu với điều kiện lọc tương ứng, 
    ví dụ \texttt{db.schedule.find(\{ userID: userID, criteria \})}, và nhận về 
    \texttt{FilteredList<Schedule>}.
    \item Danh sách đã lọc được trả lại cho \texttt{Service} rồi cho \texttt{Controller} dưới dạng \texttt{FilteredSchedule}.
    \item \texttt{Controller} hiển thị kết quả cho sinh viên thông qua lời gọi \\
    \texttt{displayFilteredSchedule()}, cho phép sinh viên xem lịch học đã được lọc theo tuần hoặc môn mong muốn.
\end{itemize}

\newpage